%%%%%%%%%%%%%%%%%%%%%%%%%%%%%%%%%%%%%%%%%%%%%%%%%%%%%%%%%%%%%%%%%%
%%%%%%%% ICML 2012 EXAMPLE LATEX SUBMISSION FILE %%%%%%%%%%%%%%%%%
%%%%%%%%%%%%%%%%%%%%%%%%%%%%%%%%%%%%%%%%%%%%%%%%%%%%%%%%%%%%%%%%%%

% Use the following line _only_ if you're still using LaTeX 2.09.
%\documentstyle[icml2012,epsf,natbib]{article}
% If you rely on Latex2e packages, like most moden people use this:
\documentclass{article}

% For figures
\usepackage{graphicx} % more modern
%\usepackage{epsfig} % less modern
\usepackage{subfigure} 
\usepackage{listings}

% For citations
\usepackage{natbib}

% For algorithms
\usepackage{algorithm}
\usepackage{algorithmic}

% As of 2011, we use the hyperref package to produce hyperlinks in the
% resulting PDF.  If this breaks your system, please commend out the
% following usepackage line and replace \usepackage{icml2012} with
% \usepackage[nohyperref]{icml2012} above.
\usepackage{hyperref}

% Packages hyperref and algorithmic misbehave sometimes.  We can fix
% this with the following command.
\newcommand{\theHalgorithm}{\arabic{algorithm}}

% Employ the following version of the ``usepackage'' statement for
% submitting the draft version of the paper for review.  This will set
% the note in the first column to ``Under review.  Do not distribute.''
%\usepackage{icml2012} 
% Employ this version of the ``usepackage'' statement after the paper has
% been accepted, when creating the final version.  This will set the
% note in the first column to ``Appearing in''
\usepackage[accepted]{icml2012}


% The \icmltitle you define below is probably too long as a header.
% Therefore, a short form for the running title is supplied here:
\icmltitlerunning{Weather Severity Prediction}

\begin{document} 

\twocolumn[
\icmltitle{Weather Severity Prediction Model \\ Project Proposal}

% It is OKAY to include author information, even for blind
% submissions: the style file will automatically remove it for you
% unless you've provided the [accepted] option to the icml2012
% package.
\icmlauthor{Alex Kopp}{ark236@cornell.edu}
\icmladdress{Cornell Tech, New York, NY}
\icmlauthor{Andrew Li}{al543@cornell.edu}
\icmladdress{Cornell Tech, New York, NY}

% You may provide any keywords that you 
% find helpful for describing your paper; these are used to populate 
% the "keywords" metadata in the PDF but will not be shown in the document
\icmlkeywords{Weather Severity Prediction, Proposal, Google, National Weather Service}

\vskip 0.3in
]

%\begin{abstract} 
%ICML 2012 full paper submissions are due February 24, 2012. Reviewing will
%be blind to the identities of the authors, and therefore identifying
%information must not appear in any way in papers submitted for review. Submissions must be in %PDF, 8 page length limit.
%\end{abstract} 
\section{Abstract}
\label{abstract}

We propose a supervised method for predicting the severity of an upcoming storm event using the free-text descriptions and instructions inside public weather alerts issued by the US National Weather Service. In particular, we show that given an alert structured in the Common Alerting Protocol (CAP) format, we can we can relate certain keywords and patterns to the severity of the storm by predicting a severity metric based on a combination of injuries, deaths, and property damage. This measure may be used to improve alert content and public awareness of an upcoming storm.

\section{Introduction}
\label{introduction}
The US National Weather Service issues weather alerts in the event of an upcoming severe weather event. These alerts trigger alerting systems that can launch news, internet, and mobile messages to warn the public of threats to life and property. These alerts however do not have a structured measure to indicate the severity of the weather event. As a result, a storm that passes by and causes little or no damage may have the same Severe Thunderstorm Warning as the 2012 derecho that killed 22 people across seven states and left 4.2 million without power for an extended period.

The goal of this project is to build a machine learning model that can predict the severity of an upcoming storm event based on the free-text descriptions and instructions inside the public alerts. 

\section{Research Questions} 
\label{researchquestions}
\begin{enumerate}
\item
Is there a relationship between certain keywords or patterns in the public alerts text descriptions and the resulting impact of a weather event?
\item 
Can we come up with a useful structured measure to indicate the severity of an upcoming weather alert that the NWS can use to improve alert content?
\end{enumerate}
\section{Related Work}
\label{related}
Although somewhat dated and mainly about classification, there is valuable information in \cite{JoachimsNLP}. Specifically, we are considering the use of \textit{term-frequence inverse document frequency (TFIDF)} instead of pure frequency counts when dealing with text processing.

According to \cite{sklearn-flow}, the best algorithm in the Scikit-Learn package, given our relatively large amount of data, is the stochastic gradient descent regressor. Other algorithms that may fit our needs include ElasticNet Lasso and Ridge Regression SVR.

\section{Data Sets}
\label{datasets}

For this project, we already have access to two datasets (discussed below). While we would like to have additional information on topics such as school closings and power outages, it does not appear as if this data is readily available. As such, we will work with what we have.

\subsection{National Weather Service Alert Database}
Google has given us access to a database of hundreds of thousands of alerts issued by the National Weather Service. The alerts span from November 2011 to February 2013 and are using the Common Alert Protocol (CAP) format. Useful information can be found inside each alert including geographic locations, dates/times, and free text describing the alert. An example CAP Report taken from the National Weather Service's website is displayed in figure \ref{example-alert} (end of document).

\begin{figure*}[t!]
\vskip 0 in
\begin{center}
\lstset{language=XML, breaklines=true, tabsize=1, basicstyle=\footnotesize, morekeywords={encoding, alert, sent, status, msgType, scope, note, info, category, event, urgency, severity,certainty, effective, expires, senderName, headline, description, instruction, area, areaDesc, polygon, geocode, valueName, value}}
\begin{lstlisting}
<alert xmlns="urn:oasis:names:tc:emergency:cap:1.1">
...
   <sent>2013-03-11T20:23:00-05:00</sent>
   <status>Actual</status>
   <msgType>Alert</msgType>
   <scope>Public</scope>
   <note>Alert for Hancock; Pearl River (Mississippi) Issued by the National Weather Service</note>
   <info>
      <category>Met</category>
      <event>Flood Warning</event>
      <urgency>Expected</urgency>
      <severity>Moderate</severity>
      <certainty>Likely</certainty>
...
      <effective>2013-03-11T20:23:00-05:00</effective>
      <expires>2013-03-13T02:23:00-05:00</expires>
      <senderName>NWS New Orleans (Southeastern Louisiana)</senderName>
      <headline>Flood Warning issued March 11 at 8:23PM CDT until further notice by NWS New Orleans</headline>
      <description>...THE FLOOD WARNING CONTINUES FOR THE FOLLOWING RIVERS IN
LOUISIANA...MISSISSIPPI...
THE PEARL RIVER NEAR BOGALUSA AFFECTING ST. TAMMANY...WASHINGTON...
HANCOCK AND PEARL RIVER COUNTIES/PARISHES
THE PEARL RIVER NEAR PEARL RIVER AFFECTING ST. TAMMANY...HANCOCK AND
PEARL RIVER COUNTIES/PARISHES
THE FLOOD WARNING CONTINUES FOR
THE PEARL RIVER NEAR PEARL RIVER.
* UNTIL FURTHER NOTICE.
* AT 7:00 PM MONDAY THE STAGE WAS 14.1 FEET.
* MINOR FLOODING IS OCCURRING AND MINOR FLOODING IS FORECAST.
* FLOOD STAGE IS 14.0 FEET.
* FORECAST...THE RIVER IS EXPECTED TO TO FALL TO A STAGE AT OR NEAR
12.0 FEET ON EARLY THURSDAY MORNING MARCH 14TH THEN RISE TO A STAGE
NEAR 15.0 FEET ON WEDNESDAY MARCH 20TH.
* IMPACT...AT 14.0 FEET...SECONDARY ROADS TO THE RIVER AND THROUGHOUT
HONEY ISLAND SWAMP ARE INUNDATED. PROPERTY IN THE VICINITY OF THE
GAGE IS FLOODED THREATENING ABOUT 20 HOMES ALONG THE LEFT BANK.
* IMPACT...AT 13.0 FEET...THE EAST AND WEST CHANNELS OF THE RIVER WILL
BEGIN TO MERGE. HONEY ISLAND SWAMP TRAILS WILL BE UNDER WATER AS
INUNDATION OF THE SWAMP BEGINS.</description>
      <instruction>FORECAST CRESTS ARE BASED UPON RAINFALL THAT HAS OCCURRED ALONG WITH
ANTICIPATED RAIN FOR THE NEXT 12 HOURS. ADJUSTMENTS TO THE FORECASTS
WILL BE MADE IF ADDITIONAL HEAVY RAINFALL OCCURS.
DO NOT DRIVE CARS THROUGH FLOODED AREAS. REMEMBER...TWO FEET OF
RUSHING WATER CAN CARRY AWAY MOST VEHICLES INCLUDING PICKUPS. TURN
AROUND AND DON`T DROWN.
A FOLLOWUP PRODUCT WILL BE ISSUED LATER. STAY TUNED TO NOAA WEATHER
RADIO...LOCAL TV AND RADIO STATIONS...OR YOUR CABLE PROVIDER...FOR
THE LATEST INFORMATION. THE LATEST GRAPHICAL HYDROLOGIC INFORMATION
CAN ALSO BE FOUND AT WEATHER.GOV.</instruction>
...
      <area>
         <areaDesc>Hancock; Pearl River</areaDesc>
         <polygon />
         <geocode>
            <valueName>FIPS6</valueName>
            <value>028045</value>
         </geocode>
         <geocode>
            <valueName>FIPS6</valueName>
            <value>028109</value>
         </geocode>
...
      </area>
   </info>
</alert>
\end{lstlisting}
\caption{Example National Weather Service Alert from \url{http://alerts.weather.gov/cap/us.php?x=1}}
\label{example-alert}
\end{center}
\vskip -0.2in
\end{figure*} 

\subsection{NOAA Storm Events Database}

The Storm Events Database is a collection of storm reports maintained by the National Oceanic and Atmospheric Administration (NOAA) \cite{stormevents}. Each report documents a particular weather-related event and includes valuable information such as geographic locations, dates/times, injuries, deaths, and property damage. Table \ref{sample-storm-event} displays an example entry from this database.

\begin{table}[t]
\caption{Example entry from NOAA's Storm Event Database. Bolded fields are the ones that we believe are most important to the project.}
\label{sample-storm-event}
\vskip 0.15in
\begin{center}
\begin{small}
\begin{sc}
\begin{tabular}{ll}
\hline
\abovespace\belowspace
Field Name & Value \\
\hline
\abovespace
last\textunderscore date\textunderscore modified    & 05/16/2007 12:31:12 \\
last\textunderscore date\textunderscore certified & 05/26/2007 14:06:08 \\
episode\textunderscore id    & 3749 \\
event\textunderscore id    & 22222 \\
\textbf{state}     & CALIFORNIA \\
\textbf{state\textunderscore fips}      & 6 \\
year      & 2007 \\
month\textunderscore name      & March \\
\textbf{event\textunderscore type}      & Funnel Cloud \\
cz\textunderscore type      & C \\
\textbf{cz\textunderscore fips}      & 73 \\
\textbf{cz\textunderscore name}      & SAN DIEGO \\
wfo      & SGX \\
\textbf{begin\textunderscore date\textunderscore time}      & 03/27/2007 15:03:00 \\
\textbf{cz\textunderscore timezone}      & PST-8 \\
\textbf{end\textunderscore date\textunderscore time}      & 03/27/2007 15:03:00 \\
\textbf{injuries\textunderscore direct}      & 0 \\
\textbf{injuries\textunderscore indirect}      & 0 \\
\textbf{deaths\textunderscore direct}      & 0 \\
\textbf{deaths\textunderscore indirect}      & 0 \\
\textbf{damage\textunderscore property}      & 0.00K \\
damage\textunderscore crops      & 0.00K \\
source      & Trained Spotter \\
magnitude      & \textit{(null)} \\
magnitude\textunderscore type      & \textit{(null)} \\
flood\textunderscore cause      & \textit{(null)} \\
category      & \textit{(null)} \\
tor\textunderscore f\textunderscore scale      & \textit{(null)} \\
tor\textunderscore length      & \textit{(null)} \\
tor\textunderscore width      & \textit{(null)} \\
tor\textunderscore other\textunderscore wfo      & \textit{(null)} \\
tor\textunderscore other\textunderscore cz\textunderscore state      & \textit{(null)} \\
tor\textunderscore other\textunderscore cz\textunderscore fips      & \textit{(null)} \\
\belowspace
tor\textunderscore other\textunderscore cz\textunderscore name      & \textit{(null)} \\
\hline
\end{tabular}
\end{sc}
\end{small}
\end{center}
\vskip -0.1in
\end{table}

\section{Technology Summary}
\label{technologysummary}

\subsection{Python 2.7}
\subsubsection{NumPy}
NumPy is a highly efficient and robust scientific computing Python package \cite{numpy}. We will use NumPy to store our large data sets 
\subsubsection{NWS-CAP-Parser}
NWS-CAP-Parser is a Python class that parses the XML of a Common Alerting Protocol (CAP) message issued by the National Weather Service (NWS)\cite{nws-cap-parser}.

\subsubsection{Natural Language Toolkit (NLTK)}
NLTK is a leading platform for building Python programs to work with human language data \cite{nltk}. We will be using it extensively for its tokenization, stopword removal, stemming, and n-gram functions.

\subsubsection{Scikit-Learn}
Scikit-learn is a simple and efficient machine learning package for Python \cite{scikit-learn}. It will be used to train various models after the feature engineering and extraction stages are complete.

\subsection{Google Compute Engine}
Google Compute Engine allows large-scale computing workloads on the same infrastructure that runs Google Search, Gmail and Ads \cite{compute-engine}. Due to the size of the data sets that will be discussed in section \ref{datasets}, and the fact that our work may be used in their products, Google was kind enough to donate ample computing resources via this new service.

\section{Proposed Milestones}
\label{milestones}

\subsection*{Week 1 (02/25 - 03/03)}
Draft formal project proposal\\
02/28 - Project Proposal Due

\subsection*{Week 2 (03/04 - 03/10)}
Match NWS alerts with events in Storm Event Database.

\subsection*{Week 3 (03/11 - 03/17)}
03/12 - Revised Project Proposal Due\\
\\
Extract features from alerts. We intend on using most of the fields within the alert as a separate feature. The free text fields will likely be tokenized, have stopwords removed, stemmed, and have bigrams generated. 

\subsection*{Week 4 (03/18 - 03/24)}
Spring Break\\
\\
Continue extracting features from alerts. Depending on the total number of alerts and time constraints, we may opt to go back and extract trigrams instead of bigrams.

\subsection*{Week 5 (03/25 - 03/31)}
03/27 - Monthly Meeting with Project Advisor (Alice)\\
\\
Build property damage model and test. We want to see if we can make a model that says "given a CAP alert, predict the amount of property damage that the storm will cause"

\subsection*{Week 6 (04/01 - 04/07)}
Build death prediction model and test. Similar to the damage model, we would like to predict the number of deaths that will result from the storm.

\subsection*{Week 7 (04/08 - 04/14)}
04/08 - Monthly Meeting with Project Advisor (Alice)\\
\\
Build injury prediction model and test. Again, similar to the damage model, one of the factors that goes into a storm's severity is the number of injuries.

\subsection*{Week 8 (04/15 - 04/21)}
Build framework that takes a CAP alert, extracts features, runs them through three prediction models, and computes a final storm severity.

\subsection*{Week 9 (04/22 - 04/28)}
Gather all results and begin drafting final report.

\subsection*{Week 10 (04/29 - 05/05)}
05/02 - Polished Draft of Project Report Due\\
\\
Start polishing the final report.

\subsection*{Week 11 (05/06 - 05/12)}
05/10 - Final Project Report Due\\
\\
Place finishing touches on the final project report.

\bibliography{Proposal}
\bibliographystyle{icml2012}

\end{document} 


% This document was modified from the file originally made available by
% Pat Langley and Andrea Danyluk for ICML-2K. This version was
% created by Lise Getoor and Tobias Scheffer, it was slightly modified  
% from the 2010 version by Thorsten Joachims & Johannes Fuernkranz, 
% slightly modified from the 2009 version by Kiri Wagstaff and 
% Sam Roweis's 2008 version, which is slightly modified from 
% Prasad Tadepalli's 2007 version which is a lightly 
% changed version of the previous year's version by Andrew Moore, 
% which was in turn edited from those of Kristian Kersting and 
% Codrina Lauth. Alex Smola contributed to the algorithmic style files.  


