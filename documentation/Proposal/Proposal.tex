%%%%%%%%%%%%%%%%%%%%%%%%%%%%%%%%%%%%%%%%%%%%%%%%%%%%%%%%%%%%%%%%%%
%%%%%%%% ICML 2012 EXAMPLE LATEX SUBMISSION FILE %%%%%%%%%%%%%%%%%
%%%%%%%%%%%%%%%%%%%%%%%%%%%%%%%%%%%%%%%%%%%%%%%%%%%%%%%%%%%%%%%%%%

% Use the following line _only_ if you're still using LaTeX 2.09.
%\documentstyle[icml2012,epsf,natbib]{article}
% If you rely on Latex2e packages, like most moden people use this:
\documentclass{article}

% For figures
\usepackage{graphicx} % more modern
%\usepackage{epsfig} % less modern
\usepackage{subfigure} 

% For citations
\usepackage{natbib}

% For algorithms
\usepackage{algorithm}
\usepackage{algorithmic}

% As of 2011, we use the hyperref package to produce hyperlinks in the
% resulting PDF.  If this breaks your system, please commend out the
% following usepackage line and replace \usepackage{icml2012} with
% \usepackage[nohyperref]{icml2012} above.
\usepackage{hyperref}

% Packages hyperref and algorithmic misbehave sometimes.  We can fix
% this with the following command.
\newcommand{\theHalgorithm}{\arabic{algorithm}}

% Employ the following version of the ``usepackage'' statement for
% submitting the draft version of the paper for review.  This will set
% the note in the first column to ``Under review.  Do not distribute.''
%\usepackage{icml2012} 
% Employ this version of the ``usepackage'' statement after the paper has
% been accepted, when creating the final version.  This will set the
% note in the first column to ``Appearing in''
\usepackage[accepted]{icml2012}


% The \icmltitle you define below is probably too long as a header.
% Therefore, a short form for the running title is supplied here:
\icmltitlerunning{Weather Severity Prediction}

\begin{document} 

\twocolumn[
\icmltitle{Weather Severity Prediction Model \\ Project Proposal}

% It is OKAY to include author information, even for blind
% submissions: the style file will automatically remove it for you
% unless you've provided the [accepted] option to the icml2012
% package.
\icmlauthor{Alex Kopp}{ark236@cornell.edu}
\icmladdress{Cornell Tech, New York, NY}
\icmlauthor{Andrew Li}{al543@cornell.edu}
\icmladdress{Cornell Tech, New York, NY}

% You may provide any keywords that you 
% find helpful for describing your paper; these are used to populate 
% the "keywords" metadata in the PDF but will not be shown in the document
\icmlkeywords{Weather Severity Prediction, Proposal, Google, National Weather Service}

\vskip 0.3in
]

%\begin{abstract} 
%ICML 2012 full paper submissions are due February 24, 2012. Reviewing will
%be blind to the identities of the authors, and therefore identifying
%information must not appear in any way in papers submitted for review. Submissions must be in %PDF, 8 page length limit.
%\end{abstract} 

\section{Introduction}
\label{introduction}

To be completed...

\section{Research Questions} 
\label{researchquestions}

To be completed...

\section{Related Work}
\label{related}

To be completed...

\section{Data Sets}
\label{datasets}

For this project, we already have access to two datasets (discussed below). While we would like to have additional information on topics such as school closings and power outages, it does not appear as if this data is readily available. As such, we will work with what we have.

\subsection{National Weather Service Alert Database}
Google has given us access to a database of hundreds of thousands of alerts issued by the National Weather Service. The alerts span from November 2011 to February 2013 and are using the Common Alert Protocol (CAP) format. Useful information can be found inside each alert including geographic locations, dates/times, and free text describing the alert.

\subsection{NOAA Storm Events Database}

The Storm Events Database is a collection of storm reports maintained by the National Oceanic and Atmospheric Administration (NOAA) \cite{stormevents}. Each report documents a particular weather-related event and includes valuable information such as geographic locations, dates/times, injuries, deaths, and property damage.

\section{Technology Summary}
\label{technologysummary}

\subsection{Python 2.7}
\subsubsection{NumPy}
NumPy is a highly efficient and robust scientific computing Python package \cite{numpy}. We will use NumPy to store our large data sets 
\subsubsection{NWS-CAP-Parser}
NWS-CAP-Parser is a Python class that parses the XML of a Common Alerting Protocol (CAP) message issued by the National Weather Service (NWS)\cite{nws-cap-parser}.

\subsubsection{Natural Language Toolkit (NLTK)}
NLTK is a leading platform for building Python programs to work with human language data \cite{nltk}. We will be using it extensively for its tokenization, stopword removal, stemming, and n-gram functions.

\subsubsection{Scikit-Learn}
Scikit-learn is a simple and efficient machine learning package for Python \cite{scikit-learn}. It will be used to train various models after the feature engineering and extraction stages are complete.

\subsection{Google Compute Engine}
Google Compute Engine allows large-scale computing workloads on the same infrastructure that runs Google Search, Gmail and Ads \cite{compute-engine}. Due to the size of the data sets that will be discussed in section \ref{datasets}, and the fact that our work may be used in their products, Google was kind enough to donate ample computing resources via this new service.

\section{Proposed Milestones}
\label{milestones}

\subsection*{Week 1 (02/25 - 03/03)}
Draft formal project proposal\\
02/28 - Project Proposal Due

\subsection*{Week 2 (03/04 - 03/10)}
Match NWS alerts with events in Storm Event Database.

\subsection*{Week 3 (03/11 - 03/17)}
03/11 - Monthly Meeting with Project Advisor (Alice)\\
03/12 - Revised Project Proposal Due\\
\\
Extract features from alerts. We intend on using most of the fields within the alert as a separate feature. The free text fields will likely be tokenized, have stopwords removed, stemmed, and have bigrams generated. 

\subsection*{Week 4 (03/18 - 03/24)}
Spring Break\\
\\
Continue extracting features from alerts. Depending on the total number of alerts and time constraints, we may opt to go back and extract trigrams instead of bigrams.

\subsection*{Week 5 (03/25 - 03/31)}
Build property damage model and test. We want to see if we can make a model that says "given a CAP alert, predict the amount of property damage that the storm will cause"

\subsection*{Week 6 (04/01 - 04/07)}
Build death prediction model and test. Similar to the damage model, we would like to predict the number of deaths that will result from the storm.

\subsection*{Week 7 (04/08 - 04/14)}
04/08 - Monthly Meeting with Project Advisor (Alice)\\
\\
Build injury prediction model and test. Again, similar to the damage model, one of the factors that goes into a storm's severity is the number of injuries.

\subsection*{Week 8 (04/15 - 04/21)}
Build framework that takes a CAP alert, extracts features, runs them through three prediction models, and computes a final storm severity.

\subsection*{Week 9 (04/22 - 04/28)}
Gather all results and begin drafting final report.

\subsection*{Week 10 (04/29 - 05/05)}
05/02 - Polished Draft of Project Report Due\\
\\
Start polishing the final report.

\subsection*{Week 11 (05/06 - 05/12)}
05/10 - Final Project Report Due\\
\\
Place finishing touches on the final project report.

\bibliography{Proposal}
\bibliographystyle{icml2012}

\end{document} 


% This document was modified from the file originally made available by
% Pat Langley and Andrea Danyluk for ICML-2K. This version was
% created by Lise Getoor and Tobias Scheffer, it was slightly modified  
% from the 2010 version by Thorsten Joachims & Johannes Fuernkranz, 
% slightly modified from the 2009 version by Kiri Wagstaff and 
% Sam Roweis's 2008 version, which is slightly modified from 
% Prasad Tadepalli's 2007 version which is a lightly 
% changed version of the previous year's version by Andrew Moore, 
% which was in turn edited from those of Kristian Kersting and 
% Codrina Lauth. Alex Smola contributed to the algorithmic style files.  


